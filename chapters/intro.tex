\chapter{Introduction}

\label{Chapter01} % For referencing the chapter elsewhere, use \ref{Chapter1} 

\section{Motivation}

\subsection{What is the brain and why do we study it?}
Humans have been fascinated by their own brains for centuries. The brain is the
seat of our memories, thoughts, and perception; it is within the brain that our
complex behaviours are generated. However, we are still far from understanding
its workings. We know that the brain is composed of a very large number of
specialised cells called neurons, which are interconnected and form a vast
network. We know that neurons process our perceptual inputs and generate
behaviour by passing information through this network, and we know that even a
slight disruption of this network can have profound consequences on our lives
and on the way we experience the world, as pathological conditions as diverse as
Parkinson's disease, aphasia and schizophrenia are caused by specific changes
that occur within the nervous system. Nevertheless, making sense of all of this
knowledge, being able to interpret it and make predictions about future
behaviour using one single model is still far beyond our reach.

Owing to continual technological advances, our knowledge of the brain has
greatly advanced in the last few decades. We are now able to watch, record and
alter the activity of single neurons, identify their interconnections and study
how these relate to behaviour or how they are affected in brain disorders. In
spite of this, treatment possibilities for many disorders remain limited and we
still struggle at devising targeted interventions. There is, nevertheless, a
general feeling among many in the field that we are currently on the cusp of a
new big leap in our understanding of the brain and several large research
initiatives focused on the brain have been recently unveiled to make this leap
possible\footnote{Human Brain Project (\url{https://www.humanbrainproject.eu}),
BRAIN Initiative (\url{https://www.braininitiative.nih.gov}), Brain/MINDS
(\url{http://brainminds.jp/en}), among others.}. These have complementary goals:
Some focus on developing new techniques to study the structure and function of
the brain, other on analytical procedures and data storage or on mapping the
brain of a specific species in great detail.

During my doctoral studies, I focused on studying the activity and structure of
the rodent brain using a medical imaging technique called magnetic resonance
imaging (MRI), and in the following sections I will briefly introduce the type
of questions that I have addressed in my research. 

\subsection{How do we measure brain activity?}

MRI is a medical imaging technique that is capable of producing images of the whole
brain. It does so by making use of magnetic properties of hydrogen nuclei
present throughout the body in the form of water molecules. The technique is
non-invasive and its images have a very good contrast between soft tissues,
such as between a tumour and its surroundings or between grey and white matter
of the brain. These properties have made MRI popular among clinicians and the
technique has become ubiquitous in modern day clinical practice
\parencite{westbrook2011}.

However, neuroscientists are not interested only in the structure of brain; they
are just as much interested in studying its activity and function: Is region~A
activated by a sensory stimulus~B? Is the performance of a subject in task~C
affected when region~D is stimulated or inhibited? Do people with disease~E show
reduced activity in region~F?

A classical approach to measuring the activity of a neuron is to implant a
microelectrode into the brain and take advantage of how changes in the electric
potential across the neuron's membrane encode the state of the neuron
\parencite{hubel1959, adrian1928}. However, this technique is invasive and
records the activity of a single neuron. Another technique called
electroencephalography (EEG) records the combined electrical activity of neurons
at the surface of the skull. While this technique is non-invasive, its spatial
resolution and sensitivity to neurons that are located deeper in the brain are
limited \parencite{jackson2014}. Currently, the most popular technique to study
brain function is based on the following observation: When neurons in a
particular region of the brain are active, the flow of oxygenated blood into
that region increases in order to deliver the fuel required by the active
neurons, and the change in the balance in concentration between oxygenated and
deoxygenated blood results in an increase of the MR signal coming from that
particular area \parencite{ogawa1990}. Therefore, in order to identify regions
of high or low neuronal activity, we need to repeatedly and in short intervals
acquire MR images of a subject's brain and track changes in MR signal across
different brain structures. This approach was further developed and resulted in
the introduction of blood-oxygen-level dependent functional MRI (BOLD fMRI) in
the early 1990s \parencite{bandettini1992,kwong1992}. In fMRI, the brain is
divided into small cubes called voxels and one can infer the activation time
series for each of these voxels individually. The spatial resolution of the
method is therefore better than that of EEG and now reaches submillimeter
values. However, its sampling frequency is relatively low at around one sample
per second, and it can by no means measure the activity of a single neuron: Its
signal is a proxy for the combined activity of all neurons in the selected
voxel convolved with the complex hemodynamic response.

Functional MRI has quickly become popular among neuroscientists, especially
because of its non-invasiveness and whole-brain coverage. A typical fMRI
experimental paradigm aims at identifying associations between sensory
stimulation or task performance and increased neuronal activity in specific
brain regions \parencite{friston1994}. Periods of activity are usually
interleaved with periods of rest during which the subject lies in an MRI scanner
with no overt task or sensory stimulation. By comparing the activity in
individual brain regions in all segments of the acquisition, researchers can
identify those regions that show statistically significant increases when the
subject is performing the task compared to the resting baseline. A large number
evoked activations were studied in this way and a number task- or
sensory-specific networks of co-activating regions have been identified.  These
studies have greatly advanced our knowledge of the functional organization of
the brain \parencite{poldrack2015a}.

\subsection{What does the brain do when we are at rest?}

The type of experiments described in the preceding paragraph paints a very
``reflexive'' picture of the brain: It sits around not doing much unless it is
externally stimulated or unless it is involved in an externally directed task.
However, there are many observations that contradict such a view of brain
function and instead suggest a brain that is constantly buzzing with activity
\parencite{raichle2010}. One of these observations is the relatively high
energy consumption of the brain that only fractionally increases during task
performance \parencite{raichle2015}. Moreover, spontaneous activity is clearly
visible on EEG and fMRI scans acquired during periods of rest and this activity
shows signs of non-random organization across brain regions which goes beyond
simple monosynaptic axonal connectivity \parencite{fox2007}.

If the ``reflexive'' function of the brain constitutes only a small fraction of
its activity, how can we uncover and interpret the intrinsic activity? This has
been the object of research within the fMRI community ever since the first
description of non-random organization of the resting-state fMRI (rsfMRI)
signal within the human motor cortex \parencite{biswal1995}. An important step
in this direction has been the observation that networks of regions whose
activity time series are correlated at rest correspond to networks of
co-activated regions elicited during task performance \parencite{smith2009,
crossley2013}. Several explanations for this phenomenon have been put forward,
including learning consolidation, preparation for task performance and
synchronization based on Hebbian-style reinforcement mechanisms, although the
evidence for the time being remains inconclusive \parencite{power2014}. The
observation of specifically organized brain activity at rest has nevertheless
marked a paradigm shift in neuroimaging and there has been a continual increase
in the application of rsfMRI to the study of brain organization
\parencite{snyder2012}. 

\subsection{What is a connectome? How is it affected in brain disorders? Can it
help us better understand or diagnose these disorders?}

The shift from task-evoked to resting-state fMRI has brought about also a shift
in the way the fMRI signal is analysed. A growing number of studies started to
study brain's intrinsic activity from the perspective of mathematical graphs or
networks: The brain is represented as a network of interconnected functional
regions \parencite{bullmore2009}. The strength of a connection between two
regions can be expressed as the degree to which the intrinsic activity in those
two regions is similar and is commonly referred to as ``functional
connectivity'' \parencite{friston2011}. Such a representation of the brain is
also referred to as the ``connectome'' \parencite{smith2013} and it is useful
both from the data analysis perspective -- the underlying field of network
analysis has burgeoned recently, too, and it provides a large number of useful
analytical tools -- and from the interpretative perspective -- as it is closer
to the actual architecture of the brain.

Recent studies have revealed two defining characteristics of human functional
connectomes. First, their nodes can be reliably partitioned into densely
connected clusters called modules or communities, most of which typically
replicate previously described functional systems of the brain
\parencite{power2011, yeo2011}. Second, there exist a limited set of nodes
called hubs that act as integrators between these systems and that are crucial
for the coordinated involvement of multiple systems in complex behaviors
\parencite{vandenheuvel2013}.

Connectomic studies have not only provided insights into the organization of the
healthy brain, but also led to observations of altered brain connectivity in a
number of brain disorders disorders: deficient connectivity of the prefrontal
cortex in schizophrenia \parencite{anticevic2013a}, widespread over- and
underconnectivity in autism spectrum disorders \parencite{dimartino2014a}, and
many others \parencite{crossley2014,buckner2009}. This area of research, also
referred to as ``pathoconnectomics'' \parencite{rubinov2013}, has seen an
especially large following, with the objective to help diagnosis, devise
measurable biomarkers, and better understand their genetic or environmental
underpinnings \parencite{castellanos2013, khalili-mahani2017}.

However, connectomic approaches to studying and diagnosing brain disorders still
face several challenges before widespread adoption can take place. Moreover,
many of these challenges have been used to argue against the use of
neuroimaging-based findings in clinical practice \parencite{weinberger2016}. For
example, there are inconsistencies across studies in connectivity alterations
observed for a specific disorder. These disparities and potential lack of
replicability may be attributable to several factors, including clinical
heterogeneity of the patient groups, imaging parameters, analytical procedures
and inadequate correction of other confounding factors such as motion or
cross-site differences \parencite{filippi2016, marchitelli2016, power2015}. As
pathoconnectomics is still a nascent field, several of these may come into play
in a single study and therefore it is imperative that the community gives them
full attention. Current developments, such as the use of registered reports
\parencite{chambers2013} and increased data and code sharing
\parencite{poldrack2017}, may help address some of the issues regarding the
general replicability of findings, by allowing the readers to distinguish
\textit{a priori} from \textit{post hoc} analyses and to potentially re-run some
of the analyses, and by allowing the researchers to replicate their results on
comparable datasets acquired elsewhere and to test their new analytical methods
on larger datasets.

\subsection{Autism spectrum disorders}

While many brain disorders -- neurological, psychiatric and developmental --
have been studied using neuroimaging techniques, in this thesis we will focus
specifically on autism spectrum disorders (ASD). Autism is a heterogeneous
syndrome characterised by core behavioural features including deficits in social
communication and interaction, as well as restricted, repetitive patterns of
behaviour, interests and activities \parencite{association2013}. Individuals
with ASD can also present several other commorbidities, such as intellectual
disability and epilepsy \parencite{zafeiriou2007, bauman2010}. While the typical
age of diagnosis used to be around 3-4 years, current efforts aim at a
much earlier diagnosis in order to allow for early intervention
\parencite{lai2014}. Nevertheless, the disorder continues to represent a
considerable burden to affected individuals, their immediate family and public
healthcare systems. 

Although a primary and unitary aetiology for ASD has not been identified, its
high heritability has been consistently documented across a series of twin
studies \parencite{tick2016}, revealing a contribution of complex and highly
heterogeneous genetic mutations \parencite{geschwind2009, geschwind2015,
sanders2015}. Remarkably, although previously identified mutations, genetic
syndromes and de novo copy number variations (CNVs) account for about 10–20 \%
of ASD cases, none of these single known genetic causes accounts for more than
1–2 \% of cases [reviewed in \parencite{abrahams2008}], making heterogeneity a
major hallmark of the disorder \parencite{betancur2011}. Nevertheless, the
advances in identification of autism-risk genes led to renewed interest in
studying the underlying neurobiology of the disorder and several major cellular
pathways have been shown to be affected, including transcription and chromatin
regulation, synapse development, and signal transduction
\parencite{sanders2015b, kleijer2017}. 

Concurrently, the advent of non-invasive brain imaging raised hopes that the
clinical heterogeneity of ASD could be narrowed down to a smaller number of
identifiable ``imaging endophenotypes'' that could help ASD diagnosis, patient
stratification, and possibly provide clues as to the elusive aetiology of this
group of disorders \parencite{geschwind2015, gottesman2003}. While several
initial studies focused on brain overgrowth as a potential early indicator of
autism \parencite{courchesne2002, lange2015, ecker2017}, a large number of
studies have since considered functional connectivity disruptions in individuals
with ASD, following early reports of reduced brain connectivity identified using
positron-emission tomography [PET, \parencite{horwitz1988}], later corroborated
by investigations with task-based \parencite{just2004} and resting-state fMRI
\parencite{assaf2010, cherkassky2006, kennedy2008}. The extensive literature
published to date points at the presence of major functional connectivity
alterations in ASD populations, although the identified regional patterns vary
considerably across studies and patient cohorts \parencite{ameis2015,
bernhardt2016, ecker2014, ecker2015, kana2011, muller2014, vasa2016}. These
results are intriguing as they are consistent with the observation that many
autism-risk genes are involved in synapse development and function: Aberrations
in these genes might lead to perturbations of neural connectivity, which in turn
could result in aberrant large-scale functional connectivity \parencite{ecker2017}. 

\subsection{Can animal research be of any value in pathoconnectomics?}

Do connectivity disruptions cause brain disorders or are they a mere by-product
of brain dysfunction? Can we identify possible genetic contributions to this
phenomenon? What is the neural basis that underpins connectivity disruptions? Is
functional connectivity disruption associated with aberrant structural
connectivity?

Despite intensive human connectomic research, many fundamental questions about
the nature of connectivity disruptions in brain disease remain open. This may be
due to the fact that there currently exists a disconnect between the studies of
large-scale organization of the human brain with techniques such as MRI and EEG,
and the research at the level of genes, neurons and microcircuits performed in
laboratory animals. Let us consider the case of ASD: Despite some obvious
limitations in reliably modeling the full phenotypic spectrum of such a complex
developmental disorder, mouse models have played a central role in advancing our
basic mechanistic and molecular understanding of the disorder
\parencite{silverman2010a, delatorre-ubieta2016}. Therefore, a deeper
understanding of the complex pathophysiological cascade leading to aberrant
connectivity in ASD can greatly benefit from the use of the mouse, where
individual pathophysiological or phenotypic components of the spectrum can be
recreated and investigated via approaches that are either off limits or
confounded by clinical heterogeneity \parencite{nestler2010}. The use of the
mouse may also help us address some of the more technical challenges of human
connectomic research discussed above: motion can be very well controlled and
rs-fMRI measurements can be complemented by other brain activity readouts, such
as by measuring local field potentials (LFP).  Moreover, there is an increasing
number of publicly available resources with rich data about the mouse brain,
including high-resolution gene expression and neural circuitry datasets produced
by the Allen Institute for Brain
Science\footnote{\url{http://www.brain-map.org/}} \parencite{lein2007, oh2014},
which pave the way for multimodal investigations and more refined biological
interpretations of connectomic findings \parencite{konopka2017,
vandenheuvel2016}. Taken together, research in the mouse allows for combined
investigations of microscopic and macroscopic findings, which are currently
called for in the field of autism research \parencite{ecker2017}.

However, while initial rsfMRI experiments in the mouse proved promising
\parencite{sforazzini2014}, before we can readily translate the knowledge across
the two species, we need to learn more about the large-scale organization of the
mouse brain and ascertain that -- at least to some extent -- it does replicate
fundamental features of the human brain connectome, such as the presence of
functional networks and hubs and their impairment in pathological states
\parencite{vandenheuvel2016a}. Similarly, methods that are used in human
research to identify points of disruptions need to be first tested in the mouse
before they can be applied to study disease models. It is these research
questions that formed the basis of my doctoral studies and that are discussed in
the following chapters of this thesis.

\section{Structure and main contributions of the thesis}

\paragraph{Chapter~\ref{Chapter02}} We build upon initial applications of rsfMRI to the
mouse by describing large-scale functional organization of a mouse brain.
Specifically, we apply a fully-weighted network analysis (1) to map
whole-brain intrinsic functional connectivity (i.e., the functional connectome)
at a high-resolution voxel scale and (2), to spatially locate functional
connectivity hubs in the mouse brain. Analysis of this large rsfMRI dataset
revealed the presence of six distinct functional modules related to known
large-scale functional partitions of the brain, including a default-mode network
(DMN). Consistent with human studies, highly-connected functional hubs were
identified in several sub-regions of the DMN, including the anterior and
posterior cingulate and prefrontal cortices, in the thalamus, and in small foci
within well-known integrative cortical structures such as the insular and
temporal association cortices. According to their integrative role, the
identified hubs exhibited mutual preferential interconnections. These findings
highlight the presence of evolutionarily-conserved, mutually-interconnected
functional hubs in the mouse brain, and may guide future investigations of the
biological foundations of aberrant rsfMRI hub connectivity associated with brain
pathological states.

\paragraph{Chapter~\ref{Chapter03}} We apply the methods developed in the
previous study to investigate functional connectivity, its aberrations and their
potential neurobiological underpinnings in mice lacking contactin associated
protein-like 2 (\textit{Cntnap2}). Homozygous mutations in CNTNAP2, a
neurexin-related cell-adhesion protein, are strongly linked to autism and
epilepsy in humans.  Importantly, the mouse model employed in this study
recapitulates not only the high-confidence ASD mutation, but also many
behavioural and neurobiological traits seen in human carriers of the same
mutation. Moreover, common genetic variants in CNTNAP2 were recently described
to be associated with impaired frontal lobe connectivity in humans, allowing
therefore for a direct comparison of our investigations with findings in humans.
Here we used rsfMRI to show that homozygous mice lacking \textit{Cntnap2}
exhibit reduced long-range and local functional connectivity in prefrontal and
midline brain ``connectivity hubs.''Long-range rsfMRI connectivity impairments
affected heteromodal cortical regions and were prominent between
fronto-posterior components of the mouse default-mode network, an effect that
was associated with reduced social investigation, a core ``autism trait'' in
mice. Notably, viral tracing revealed reduced frequency of prefrontal-projecting
neural clusters in the cingulate cortex of \textit{Cntnap2}$^{-/-}$ mutants,
suggesting a possible contribution of defective mesoscale axonal wiring to the
observed functional impairments. Macroscale cortico-cortical white-matter
organization appeared to be otherwise preserved in these animals. These findings
reveal a key contribution of ASD-associated gene CNTNAP2 in modulating
macroscale functional connectivity, and suggest that homozygous loss-of-function
mutations in this gene may predispose to neurodevelopmental disorders and autism
through a selective dysregulation of connectivity in integrative prefrontal
areas.

\paragraph{Chapter~\ref{Chapter04}} We discuss recent progress in mouse brain
connectivity mapping via rsfMRI in the context of autism connectivity research
and show its growing potential to generate and test mechanistic hypotheses about
the elusive origin and significance of connectional aberrations observed in
autism. Furthermore, we describe initial examples of how the approach can be
employed to establish causal links between ASD-related mutations, developmental
processes, and brain connectional architecture.

\paragraph{Chapter~\ref{Chapter05}} In this final chapter we summarize the
findings of this thesis and discuss future directions.
