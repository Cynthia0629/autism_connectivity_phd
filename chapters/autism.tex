\chapter{Mouse imaging and the autism connectivity chaos}

\label{Chapter04}

\begin{quote}
    This chapter has been published as:

    \longfullcite{liska2016}
\end{quote}

\section{The connectivity theory of autism: open questions and controversies}

Autism is a heterogeneous syndrome characterised by core behavioural features
including deficits in social communication and interaction, as well as
restricted, repetitive patterns of behaviour and interests
\parencite{association2013}.  Although a primary and unitary aetiology for
autism spectrum disorder (ASD) has not been identified, its high heritability
has been consistently documented, revealing a contribution of complex and highly
heterogeneous genetic mutations \parencite{geschwind2009, geschwind2015,
sanders2015}. Remarkably, although previously identified mutations, genetic
syndromes and de novo copy number variations (CNVs) account for about 10–20 \%
of ASD cases, none of these single known genetic causes accounts for more than
1–2 \% of cases [reviewed in \parencite{abrahams2008}]. The phenotypic
expression (i.e., “penetrance”) of these genetic components is also highly
variable, ranging from fully penetrant point mutations to polygenic forms with
multiple gene–gene and gene–environment interactions. Remarkable variability
exists also in the extent of cognitive and behavioural abnormalities presented
by affected individuals \parencite{lai2014, chang2015, georgiades2013}, making
heterogeneity a dominant theme for this group of disorders. 

The advent of non-invasive brain imaging raised hopes that such clinical
heterogeneity could be narrowed down to a small number of identifiable “imaging
endophenotypes” that could help ASD diagnosis, patient stratification, and
possibly provide clues as to the elusive aetiology of this group of disorders.
Unfortunately, the results of imaging studies have proven overall as variable as
the clinical manifestations of ASD \parencite{ecker2015, stanfield2008}. A
notable exception to this scenario was the initial observation of reduced
connectivity between brain regions in ASD patients, a finding first reported by
Horowitz and colleagues \parencite{horwitz1988} using PET, and later
corroborated by task-based \parencite{just2004} and resting-state fMRI (rsfMRI)
studies \parencite{assaf2010, cherkassky2006, kennedy2008}, which revealed
impaired long-range synchronization in spontaneous brain activity. Together with
evidence of reduced white matter connectivity detected with MRI [reviewed in
\parencite{anagnostou2011}], these observations form the basis of the so called
“under-connectivity theory of autism” \parencite{anagnostou2011, just2012},
according to which deficient long-range communication between brain regions may
underlie ASD symptoms and pathophysiology. However, recent imaging studies have
strongly challenged this view, highlighting a much more heterogeneous picture
[see \parencite{vasa2016} for a recent review]. For example, rsfMRI mapping in a
large cohort of patients has revealed the presence of concomitant hypo- and
hyper-connectivity \parencite{dimartino2014a}, although a clear prevalence of
hypo-connected regions was apparent. Similarly, widespread hyper-connectivity
during childhood has also been recently described \parencite{keown2013,
supekar2013, uddin2013}, suggesting a possible neurodevelopmental origin for
these alterations. More recently, the hypothesis that such conflicting findings
could reflect greater inter-subject variability in ASD patients than in
neurotypical controls (i.e. idiosyncratic connectivity) has been proposed
\parencite{hahamy2015}. A putative confounding contribution of ASD-related
motion and its effect on functional connectivity readouts is also the subject of
an open controversy in the imaging community \parencite{deen2012, power2015,
power2012, pardoe2016}. 

Collectively, the extensive literature published to date points at the presence
of major functional connectivity alterations in ASD populations, although the
identified regional patterns vary considerably across studies and patient
cohorts \parencite{ameis2015, bernhardt2016, ecker2014, ecker2015, kana2011,
muller2014, vasa2016}. Despite this rapidly accumulating evidence, many
fundamental questions as to the origin and significance of connectional
alterations in ASD remain unanswered. For one, the neurophysiological
underpinnings of these connectional aberrancies are largely unknown, and a
causal etiopathological contribution of specific genetic variants to impaired
connectivity in ASD remains to be firmly established. More broadly, it is
unclear whether these abnormalities are a causative or epiphenomenal consequence
of the disease, and whether their heterogeneous expression reflects cohort
effects, different genetic aetiologies or neurodevelopmental trajectories. The
exact relationship between connectivity alterations and the severity of ASD
manifestation remains also obscure, with the vast majority of the human
neuroimaging literature being focused on high functioning ASD cohorts
\parencite{vissers2012}. 

A deeper understanding of the origin and significance of these phenomena is
greatly complicated by our very limited understanding of the neurobiological
foundations of macro-scale neuroimaging readouts commonly employed in ASD
research, such as white matter microstructural parameters [e.g. fractional
anisotropy \parencite{owen2014}], or the elusive functional couplings underlying
rsfMRI-based functional connectivity. This has left us with a major explanatory
gap between mechanistic models of brain function at the cellular and
microcircuit level, and the emergence of macroscale functional activity in
health and pathological states such as those that are observed in autism. As a
result, we are currently unable to properly interpret and back-translate
clinical evidence of aberrant connectivity into interpretable neurophysiological
events/models that can help understand, diagnose or treat these disorders. It is
also becoming apparent that a full disambiguation of the multifactorial and
complex determinants of aberrant functional connectivity in ASD can only be
obtained through the combined use of refined clinical imaging methods and
multimodal-multiscale investigational approaches that currently can only be
applied in experimental animal models.

\section{Bridging the gap: functional connectivity mapping in mouse autism
models}

The identification of several high-confidence ASD-risk genes involved in
syndromic forms of autism \parencite{sanders2015} has been paralleled by the
generation of mouse lines recapitulating human mutations. Despite predictable
limitations in reliably modelling the full phenotypic spectrum of a complex (and
possibly only human) developmental disorder like ASD, mouse models can be
harnessed to understand how genetic alterations translate into relevant changes
in cells and circuits, and ultimately to identify points of convergence for
molecular pathways, cells, circuits, and systems that may result in a deeper
understanding of the pathophysiology of ASD and related behavioural deficits
\parencite{arguello2012, nelson2015, vasa2016}. For example, molecular
investigations in ASD mouse models have been instrumental in the identification
of a limited set of molecular pathways to which ASD-involved genes seem to
converge, including, among others, synaptogenesis, synaptic function, and
neuronal translational regulation [reviewed in
\parencite{delatorre-ubieta2016}]. This effort has been accompanied by the
development of ASD-relevant behavioural phenotyping assays, primarily targeted
at social, communication and repetitive behaviours \parencite{wohr2013, kas2014,
silverman2010, homberg2016}. Interestingly, many – but not all – models showed
autism-like traits, with manifestations ranging from repetitive behaviours to
reduced social communication (ultrasonic vocalizations) and social interest
[reviewed in \parencite{ellegood2015}]. However, despite the widespread
application and high face validity of ASD behavioural phenotyping, the
significance and translational relevance of mouse behavioural alterations to
human ASD remain debated \parencite{wohr2013}, and should be extrapolated with
caution. 

Recent advances in mouse rsfMRI mapping [reviewed in \parencite{gozzi2016}]
offer the opportunity of extending mouse modelling of ASD to the investigation
of the neurobiological underpinnings and etiopathological significance of
ASD-related connectivity aberrations. Specifically, improvements in MRI imaging
hardware, together with tighter control of physiological and motion artefacts
\parencite{ferrari2012, weber2006} have led to robust and reproducible
identification of homotopic rsfMRI networks covering known cortical and
subcortical systems in the mouse by several research groups
\parencite{mechling2014, nasrallah2014, sforazzini2014, shah2016, zerbi2015}.
Interestingly, distributed networks encompassing heteromodal prefrontal and
posterior cortical regions have also been identified \parencite{sforazzini2014,
shah2016, zerbi2015}, leading to the suggestive hypothesis of the presence of
evolutionary precursors of the human salience network and default mode network
(DMN) in this species [reviewed in \parencite{gozzi2016}]. This notion is
empirically corroborated by the recent observation that
cytoarchitecturally-homologue regions such as anterior cingulate and
retrosplenial cortices \parencite{vogt2014} similarly serve as connectivity hubs
in humans and mice \parencite{cole2010, liska2015, tomasi2011}. Moreover, the
application of rsfMRI to the mouse brain comes with several important
advantages, including the possibility to use quantitative imaging modalities for
an objective endo-phenotypic characterization of ASD-related pathology
complementary to behavioural assays, and to validate its readouts with invasive
techniques that are off limits for human research, including local field
potentials (LFPs) coherence mappings \parencite{zhan2014}, local injection of
neuronal tracers \parencite{sforazzini2016}, as well as an ever-increasing array
of histopathological, stereological or immunohistochemical post-mortem analyses.

Collectively, these correspondences strongly support the use of rsfMRI as a
means to bridge research of functional connectivity aberrancies in autism across
species (man vs. mouse) and levels of inquiry (from cellular- and microscale to
meso- and macroscale, Fig.~\ref{fig:autism_fig01_bridge}), along two main
investigational routes. First, rsfMRI can be used to establish causal (rather
than associative) etiopathological contributions between specific ASD-associated
genetic variants and macroscale connectivity, thus complementing analogous
clinical research efforts using imaging genetics \parencite{rudie2012,
scott-vanzeeland2010}. One notable experimental advantage of mouse imaging with
respect to current human imaging genetic approaches is the possibility of
mapping and comparing the effect of multiple mutations (via the use of different
autism mouse models) under rigorously controlled experimental conditions, thus
reducing the confounding contribution of experimental variables that can be only
minimally controlled in human research, such as genetic and environmental
variability, age \parencite{uddin2013}, ASD-related motion, and group
differences in cognitive states \parencite{vasa2016}. The main goal of this line
of investigation is to assess whether seemingly unrelated ASD-risk mutations do
converge on a limited number of distinct functional connectivity endophenotypes.
An elegant demonstration of this approach has been recently described using
morpho-anatomical MRI. Brain-volumetric phenotypes of 26 ASD mouse models using
structural MRI methods exhibited clustering into three main groups, each with a
distinct set of concomitant changes in size across different brain regions
\parencite{ellegood2015a}. Such reduction of morpho-anatomical heterogeneity is
not surprising, given the wide (and sometimes opposing) stream of
pathophysiological alterations observed in syndromic forms of autism, which
range from basic molecular or synaptic mechanism such as protein synthesis
\parencite{geschwind2007, auerbach2011} up to homeostatic regulations of
excitatory and inhibitory neurotransmission \parencite{nelson2015}. Analogous
analyses with regards to functional connectivity phenotypes should be possible
in the future to associate basic pathophysiological traits with macroscale
connectional aberrancies. 

\begin{figure}[th] \centering
    \includegraphics[scale=0.75]{figures/autism_figure_01_bridge.png}
    \decoRule
    \caption[Mouse imaging can bridge
    the gap between microscale models of brain function, and clinical research
    of macroscale functional connectivity.]{Mouse imaging can bridge
    the gap between microscale models of brain function, and clinical research
    of macroscale functional connectivity. Mouse models provide a powerful
    reductive platform that can be employed to link etiological determinants of
    ASD, such as syndromic mutation or neurodevelopmental traits, to basic
    molecular and cellular signatures of pathology (left, top to down). However,
    until recently we have been unable to use this approach to study the
    neurobiological underpinnings of macroscale functional connectivity, owing
    to difficulty in translating models of brain function across levels of
    inquiry. This results in a major explanatory gap between clinical research
    (heavily relying on macroscale neuroimaging measures of brain function, such
    as rsfMRI) and preclinical neurobiological investigation in rodent models
    (bottom, right). The implementation of functional connectivity mapping via
    rsfMRI in the mouse (right) can bridge this gap, by permitting to causally
    relate connectional changes with basic molecular or cellular processes, and
    by permitting a direct translation of these findings from and to humans
    owing to the shared biophysical principle underlying these measurements
    [adapted from \parencite{arguello2012, anticevic2013}].}
    \label{fig:autism_fig01_bridge}
\end{figure}

A second main line of investigation is the combined use of mouse rsfMRI and
multiscale neurobiological techniques to obtain a mechanistic description of
ASD-related phenotypes and pathophysiological pathways leading to aberrant
functional connectivity. This research can include, but is not limited to, a
deeper investigation of syndromic ASD mutations associated with specific
pathological traits [e.g. Tuberous Sclerosis 2 as a key mediator of impaired
autophagy and increased synaptic density \parencite{tang2014}], and can possibly
be extended to investigate risk factors that have been also more loosely
implicated in autism. This research effort may generate crucial mechanistic
information that can be used to back-translate clinical evidence of aberrant
connectivity into interpretable neurophysiological events/models that can help
understand, diagnose or treat these disorders. A brief description of initial
steps towards these two main goals is reported in the next two sections.

\section{Functional connectivity mapping in genetic models of autism}

An outstanding question in ASD connectivity studies is whether genetic mutations
associated with syndromic forms of autism are sufficient to produce aberrant
macroscale functional connectivity. Initial mouse rsfMRI studies seem to
corroborate this hypothesis. Specifically, Haberl and colleagues have recently
investigated functional and structural connectivity in the Fmr1-/y model of
fragile X syndrome (FXS) \parencite{budimirovic2011} and described connectional
aberrations in sensory networks \parencite{haberl2015}. These included reduced
structural integrity of the corpus callosum and an increase in local
connectivity of the primary visual cortex, as probed by viral tracers, an effect
accompanied by reduced rsfMRI coupling between visual and other neighbouring
sensory cortical regions. The authors suggested that the observed decoupling
could explain sensory processing defects that are often observed in FSX patients
\parencite{boyd2010}. 

In another recent study, homozygous mice lacking the ASD-risk gene CNTNAP2
\parencite{penagarikano2011} exhibited reduced long-range and local functional
connectivity in cingulate and prefrontal regions \parencite{liska2017}, two key
heteromodal areas of the mouse brain previously characterised as functional
connectivity hubs, owing to their rich connectivity with other brain areas
\parencite{liska2015}. Interestingly, impaired antero-posterior prefrontal
connectivity between components of the mouse DMN was associated with reduced
social investigation, a behavioural measure regarded as a core “autism trait” in
mice \parencite{wohr2013}. This finding recapitulates analogous imaging results
obtained in human carriers of CNTNAP2 gene polymorphisms
\parencite{scott-vanzeeland2010}, hence providing a first example of the
translational value of this approach. This finding is consistent with the
presence of impaired GABAergic neurotransmission in these animals
\parencite{penagarikano2011}, a trait that could result in aberrant oscillatory
rhythms. It is interesting to note that analogous prefrontal hypo-connectivity
has been observed using rsfMRI in BTBR mice, an idiopathic model of autism
characterised by agenesis of the corpus callosum and by analogous
excitatory/inhibitory imbalances \parencite{sforazzini2016}.

rsfMRI mapping has also been recently carried out in a mouse model of human
15q13.3 microdeletion, a CNV associated with schizophrenia, intellectual
disability and ASD \parencite{shinawi2009}. Compared to wild-type mice, 15q13.3
mice showed widespread patterns of hyper-connectivity along the
hippocampal-prefrontal axis, a network commonly affected in schizophrenic
patients \parencite{gass2016}. Notably, Gass and colleagues also showed that
aberrant functional connectivity could be acutely rescued by pharmacological
stimulation of nicotinic acetylcholine alpha 7 receptors, in keeping with a
contribution of this mechanism to the development of schizophrenia-related
phenotypes in these mice \parencite{gass2016}. Although the phenotypic traits of
this mouse line appear to be more closely related to schizophrenia rather than
to ASD \parencite{fejgin2014}, the results of this study are important as they
show that CNVs and genetic alterations with partial penetrance to ASD could
produce divergent connectional phenotypes (e.g. hyper- and hypo-connectivity),
suggesting a plausible contribution of genetic heterogeneity to some of the
discrepant imaging findings in humans. Importantly, these initial mouse studies
argue against an artefactual (e.g. motion-driven) origin of connectivity
aberrations reported in human ASD research, because the use of light sedation in
mice along with artificial ventilation allows for the acquisition of virtually
motion-free images.

\section{Neurobiological pathways leading to aberrant functional connectivity}

A few recent studies have provided important mechanistic investigations of
ASD-relevant phenotypes associated with aberrant functional connectivity. In the
first of such studies, Zhan and colleagues \parencite{zhan2014} investigated
whether deficits in synaptic pruning, a putative pathophysiological determinant
of autism \parencite{hutsler2010}, result in impaired connectivity alterations.
To probe this hypothesis, the authors measured rsfMRI connectivity in Cx3cr1KO
mice, a mouse line characterised by microglia-dependent synaptic pruning
deficits as a results of deficient neuronal-microglia signalling
\parencite{paolicelli2011}. Synaptic pruning deficits in Cx3cr1KO were found to
be associated with long-range functional connectivity impairments, a finding
corroborated by LFPs coherence recordings in freely-behaving animals.
Interestingly, the authors also showed that impaired pruning was associated with
core mouse “autism traits”, and that long-range fronto-hippocampal connectivity
was a good predictor of social behaviour. This study is of special importance,
as it was the first to suggest a role for dysfunctional synaptic maturation in
shaping long-range functional synchronization, and to postulate a contribution
of immune system mediators to this cascade. Empirical evidence in support of
this hypothesis comes from another recent study \parencite{kim2016}, where
analogous phenotypes where observed in mice characterised by defective autophagy
in microglia, including increased synaptic density, impaired social activity,
and a trend for impaired connectivity between posterior-sensory and prefrontal
regions. Similarly, Filiano and colleagues \parencite{filiano2016} recently
showed that deficiency in interferon-$\gamma$, a key immune signalling protein, is
associated with social deficits and frontal rsfMRI hyper-connectivity in SCID
mice, thus corroborating a putative mechanistic link between immune dysfunction,
impaired social behaviour and functional connectivity. Although promising and
mechanistically relevant, these initial results should be extrapolated to autism
research with great caution, as a pathophysiological contribution of immune and
microglia deficits to ASD has yet to be unambiguously demonstrated
\parencite{estes2015}. They, however, powerfully illustrate how the combined use
of rsfMRI, mouse genetics and state-of-the-art neuro-biological approaches can
elucidate pathways leading to aberrant functional connectivity, an approach that
can be extended to investigate the role of multiple ASD-relevant
pathophysiological factors, including syndromic genetic mutations.

\section{Limitations and future perspectives}

Like any other experimental approach, mouse rsfMRI is accompanied by limitations
that should be taken into account when the approach is used to investigate the
basis of connectivity alterations in ASD. First and foremost, as mouse rsfMRI
experiments normally employ sedation to minimize stress and motion of animals
during scans, the contribution of possible genotype-dependent differences in
sensitivity to anaesthesia \parencite{petrinovic2016} should be controlled. The
fact that to date only a minority of studies \parencite{sforazzini2016,
zhan2014} have reported genotype-dependent measures of anaesthesia sensitivity
is a factor for concern, as differences in anaesthesia depth/sensitivity can
affect connectivity strength and distribution of the imaged networks
\parencite{nasrallah2014}. The impact of anaesthesia per se as a putative
modifier of intrinsic connectional architecture appears to be less of an issue,
as a large body of human and rodent research shows that, under light controlled
sedation, the regional patterns of functional correlation seem to be largely
preserved [reviewed in \parencite{gozzi2016}]. As pointed out in previous work,
a rigorous control of motion and physiological state is also of paramount
importance to obtain reliable network mapping \parencite{gozzi2016,
jonckers2015}. It should also be mentioned that, although the field is still
lacking in standardised protocols and methods that would facilitate comparison
of experimental results across studies and sites, this issue is receiving
increased attention and collaborative efforts are underway to address it.

The initial studies described here represent only the first step toward a
greater understanding of the origin and underpinnings of connectional
alterations in ASD. Future investigations are required to describe commonalities
and differences between brain functional networks in the mouse and human from
multiple points of view, including topology \parencite{sporns2016,
vandenheuvel2016a}, biological underpinnings \parencite{vandenheuvel2016,
richiardi2015, wang2015}, and functional equivalence \parencite{li2015}.
Similarly, studies of additional genetic aetiologies associated with ASDs,
covering heterogeneous pathophysiological pathways, are crucial to achieve a
deeper understanding of whether the connectional signatures are mutation
specific, or can be regarded as a generalizable phenomenon. When coupled to
analogous clinical efforts aimed at identification of connectional aberrancies
in genetically homogeneous populations [e.g. 16p11.2 deletion
\parencite{owen2014, simonsvipconsortium2012}], the method can also be used to
investigate the cellular and physiological basis of clinically-relevant
neuroimaging readouts, and, via a comparison between human and mouse imaging
findings, to obtain an assessment of the translational and construct validity of
mouse models of ASD. The developmental trajectory of these alterations could in
principle also be investigated in mouse models, although critical limitations in
the accuracy of physiological control in young mice and pups exist. 

Much of mouse ASD modelling has been so far primarily addressed at monogenic ASD
syndromes, which represent approximately 10 \% of ASDs \parencite{silverman2010,
nelson2015}. The recapitulation, in mice, of high-confidence genetic aetiologies
associated with ASD offers the opportunity to probe specific hypotheses about
circuit dysfunction and ASD pathology that can be directly extrapolated to
homologous clinical populations [e.g. 16p11.2 microdeletion \parencite{owen2014,
simonsvipconsortium2012}]. An important limitation of current ASD translational
research is its inability to reliably model “idiopathic” autism, which is the
most frequent diagnostic label for ASD-related behavioural manifestations.
Attempts to use forward genetic approaches in inbred mouse lines exhibiting
ASD-like behaviours without a specific genetic determinant have been proposed,
with the inbred BTBR mouse line probably being the most notable example in the
field \parencite{silverman2010, gogolla2014, squillace2014}. Translational
relevance of neuro-behavioural findings obtained by comparing genetically
homogeneous inbred lines like asocial BTBR and “normosocial” B6 mice is,
however, debated \parencite{dodero2013, squillace2014}. Nevertheless, novel
neuromolecular approaches and the use of induced pluripotent stem cells (iPSCs)
from patients have begun to reveal common downstream neurobiological pathways in
idiopathic forms of autism characterised by shared neuroanatomical features
[e.g. macrocephaly \parencite{marchetto2016, nicolini2015}]. Controlled
manipulation of such signalling and molecular pathways in animal models is a
foreseeable strategy that can be employed to expand our translational framework
to the investigation of macroscale brain network aberrancies in idiopathic forms
of ASD.

Finally, studies in which connectivity alterations are pharmacologically or
genetically rescued may help clarify the relevance of functional connectional
alterations to ASD pathology and its behavioural manifestations. Specifically,
if connectivity alterations are an underlying cause of observed behavioural
deficits, then behavioural phenotypic “rescue” should be accompanied by
normalised patterns of brain functional connectivity in the brain. This research
could indicate whether connectivity alterations are necessary for the expression
of ASD-related behaviours in mice, or are instead an epiphenomenal manifestation
of underlying pathophysiology, thus providing an empirical assessment of the
pathophysiological relevance of connectivity aberrancies in ASD. “Rescue”
studies may also help identify putative endo-phenotypes (complementary to
behaviour) that could serve as measurable readouts for early clinical
translation and evaluation of novel ASD treatments in genetically defined autism
syndromes \parencite{smucny2014}.

In conclusion, functional imaging of the mouse has now reached a turning point
such that accurate modelling and investigation of ASD-connectivity aberrations
is currently possible, via the use of readouts amenable to direct translation to
human research (i.e., rsfMRI). Despite caveats, in the next few years the
approach is poised to offer breakthroughs in our understanding of the
pathogenesis of ASD-related connectivity aberrancies, possibly bringing some
order to the intricate and often contradictory body of research detailing
connectional alterations in patient populations. 
