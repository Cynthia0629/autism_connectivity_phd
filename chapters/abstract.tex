Functional Magnetic Resonance Imaging (fMRI) has consistently highlighted
aberrant functional connectivity across brain regions of autism spectrum
disorder (ASD) patients. However, the manifestation and neural substrates of
these alterations are highly heterogeneous and often conflicting. Moreover,
their neurobiological underpinnings and etiopathological significance remain
largely unknown. A deeper understanding of the complex pathophysiological
cascade leading to impaired connectivity in ASD can greatly benefit from the use
of model organisms where individual pathophysiological or phenotypic components
of ASD can be recreated and investigated via approaches that are either off
limits or confounded by clinical heterogeneity.

In this work, we first describe the intrinsic organization of the mouse brain at
the macroscale as seen through resting-state fMRI (rsfMRI). The analysis of a
large rsfMRI dataset revealed the presence of six distinct functional modules
related to known brainwide functional partitions, including a homologue of the
human default-mode network (DMN). Consistent with human studies, interconnected
functional hubs were identified in several sub-regions of the DMN, in the
thalamus, and in small foci within integrative cortical structures such as the
insular and temporal association cortices.

We then study the effects of mutations in \textit{contactin associated
protein-like 2} (\textit{Cntnap2}), a neurexin-related cell-adhesion protein, on
functional connectivity.  Homozygous mutations in this gene are strongly linked
to autism and epilepsy in humans, and using rsfMRI, we showed that homozygous
mice lacking \textit{Cntnap2} exhibit aberrant functional connectivity in
prefrontal and midline functional hubs, an effect that was associated with
reduced social investigation, a core “autism trait” in mice.  Notably, viral
tracing revealed reduced frequency of prefrontal-projecting neural clusters in
the cingulate cortex of \textit{Cntnap2}$^{-/-}$ mutants, suggesting a possible
contribution of defective mesoscale axonal wiring to the observed functional
impairments.  Macroscale cortico-cortical white-matter organization appeared to
be otherwise preserved in these animals. These findings revealed a key
contribution of ASD-associated gene CNTNAP2 in modulating macroscale functional
connectivity, and suggest that homozygous loss-of-function mutations in this
gene may predispose to neurodevelopmental disorders and autism through a
selective dysregulation of connectivity in integrative prefrontal areas.

Finally, we discuss the role mouse models could play in generating and testing
mechanistic hypotheses about the elusive origin and significance of connectional
aberrations observed in autism and recent progress towards this goal.
