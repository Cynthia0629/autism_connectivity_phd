\chapter{Conclusions}

\label{Chapter05}

Resting-state functional Magnetic Resonance Imaging (rsfMRI) methods have been
extensively used to explore the intrinsic organization of the human brain. In
keeping with the common conceptualisation of many brain disorders as instances
of neuronal miswiring, rsfMRI methods are being increasingly applied to study
functional connectivity alterations associated with disease, with the goal of
improving diagnoses and providing a deeper understanding of the underlying
pathophysiology. However, despite intensive human research, many research
questions relating connectivity to brain disorders remain open. As we have
argued in the preceding chapters, this situation may stem from the fact that
there is an explanatory gap between ``macroscopic'' connectivity studies
performed in humans and ``microscopic'' research in animals, which has made it
difficult to translate models of brain function across these different levels of
enquiry. 

\section{Overview of the results}

The research presented in this work focused on bridging the explanatory gap by
causally relating connectional changes with basic molecular or cellular
processes through the application of rsfMRI to the mouse brain. The main
advantage of this method is the potential to directly translate findings to and
from humans owing to the shared biophysical principle underlying rsfMRI
measurements in both species. We describe the intrinsic functional organization
of the mouse brain at the macroscale and show that there exist at least six
distinct functional modules related to known functional partitions of the brain,
including a potential rodent homologue of the default-mode network (DMN). We
next focused on describing functional connectivity in homozygous mice lacking
the gene \textit{Cntnap2}, a mutation which is strongly associated to autism.
\textit{Cntnap2} knock-out mice exhibited reduced long-range and local
functional connectivity within prefrontal and midline brain regions.  Moreover,
long-range rsfMRI connectivity impairments strongly affected the
fronto-posterior components of the mouse default-mode network, and this effect
was associated with reduced social investigation, a core ``autism trait'' in
mice. While marcoscale cortico-cortical white-matter organization appeared to be
preserved in these animals, viral tracing revealed reduced frequency of
prefrontal-projecting neural clusters in the cingulate cortex of
\textit{Cntnap2} KO mutants, suggesting a possible contribution of defective
mesoscale axonal wiring to the observed functional and consequently behavioural
impairments. Collectively, the results described in this thesis highlight the
presence of evolutionarily-conserved networks and functional hubs in the mouse
brain.  Moreover, they also reveal a key contribution of ASD-associated gene
CNTNAP2 in modulating macroscale functional connectivity, and suggest that
homozygous loss-of-function mutations in this gene may predispose to
neurodevelopmental disorders and autism through a selective dysregulation of
connectivity in integrative prefrontal areas. 

\section{Limitations}

A limitation of the studies presented in this thesis is that  all mouse imaging
has been performed under light anaesthesia, while majority of human imaging is
performed in the awake state. The main motivation for this methodological choice
is that it facilitates rigorous control of motion and physiological state, which
are both very important for reliable connectivity mapping with fMRI. As
discussed in previous chapters, light anaesthesia does not seem to affect the
intrinsic functional architecture of the brain \parencite{gozzi2016}.
Nevertheless, genotype-specific effects of individual anaesthetics cannot be in
general ruled out. In the CNTNAP2 study, we have attempted to mitigate such
confounds by comparing mean arterial blood pressure and amplitude of cortical
BOLD signal fluctuations, two measures which had been previously shown to
correlate with anaesthesia depth, across the two experimental groups. The impact
of future studies could be greatly increased by moving towards awake mouse
rs-fMRI; however, awake rs-fMRI in this animal presents specific problems and it
is currently developed by several research groups.

Another limitation of the presented studies is that they were performed on adult
mice, while the results of human neuroimaging highlighted diverse connectional
disruptions at different points during the lifespan of individuals with ASD
\parencite{ecker2015}. Longitudinal investigations of connectivity in rodent
genetic models of autism are therefore highly warranted in order to study
whether connectivity aberrancies in these models also follow similar
developmental trajectories.

\section{Future directions}

There are several areas in which we can expect important contributions from
preclinical rsfMRI in the near future. Cell-type specific manipulations through
chemo- and optogenetics make it possible to establish causal links between the
activity of neuronal subpopulations and large-scale brain activity and behaviour
\parencite{deisseroth2015, roth2016}. The application range of these techniques
is wide and -- when coupled with fMRI \parencite{giorgi2017, grayson2016} --
they provide us with means to move beyond the description of genetic models of
autism to testing specific hypotheses about the neural drivers of macroscale
functional connectivity and their potential aberrations in autism. As an
example, an intriguing result across a large number of human studies
investigating functional connectivity disruptions in brain disorders points at
the preferential disruption of network hubs in brain disease
\parencite{crossley2014}. The conservation of functional hubs in the human and
mouse (Chapter~\ref{Chapter02}) along with the observation that functional hubs
may also be points of vulnerability in mouse models of ASD
(Chapter~\ref{Chapter03}) pave the way to further investigations into the role
of functional hubs in orchestrating the dynamics of brain function. By
inhibiting or exciting physiologically distinct neuronal populations in hub and
peripheral regions, altering therefore the balance between excitatory and
inhibitory neurons crucial to both local and long-range cortical computations
\parencite{anticevic2017, krystal2017}, we could study whether the hub regions
indeed represent entry-points for network breakdown. Such studies could also
shed light on the effects of inhibitory neuronal dysfunction and increased
excitatory/inhibitory ratio observed in individuals with ASD
\parencite{marin2012}. 

Multimodal investigations linking gene expression, structural connectivity and
functional connectivity represent another interesting line of research enabled
by advances in rodent rs-fMRI. Recent experiments show that gene expression
patterns exhibit strong correlation with functionally-coupled resting state
networks of the human cortex \parencite{richiardi2015, konopka2017, wang2015}.
However, the regulatory program leading to the establishment of network-specific
transcriptional signatures remains undefined. Uncovering such a program using
high-resolution gene expression data available for the mouse brain would
enable us to investigate -- through enrichment analyses -- the potential link 
between aberrations in autism-risk genes and connectivity
disruptions within specific functional networks.
